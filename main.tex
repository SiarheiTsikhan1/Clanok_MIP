\documentclass[sig-alternate]{article}


\usepackage[left=.7in,right=.7in,top=1in]{geometry}   
\usepackage[utf8]{inputenc}    
\usepackage[slovak]{babel}	                          
\usepackage{fancyhdr}                                 
\usepackage{setspace}				         
\usepackage{multicol}                           
\usepackage{sectsty}                              
\usepackage{balance}                               
\usepackage[raggedright]{titlesec}                   
\usepackage{titlecaps}
\usepackage{url}                                   
\usepackage{breakurl}                                
\usepackage[breaklinks]{hyperref}                     

\setlength{\columnsep}{0.8cm}                            
\setlength{\parindent}{0pt}                             
\setstretch{1}						                   
\titleformat{\section}                                     
  {\normalfont\fontsize{12pt}{14pt}\selectfont\bfseries\tolerance = 250}
  {\thesection}
  {1em}
  {\MakeUppercase}
\titleformat{\subsection}                                 
  {\normalfont\bfseries\large}
  {\thesubsection}
  {1em}
  {}


\pagestyle{fancy}                                                            
\fancyhf{}
\renewcommand{\headrulewidth}{0pt}                                             
 \lhead{\sffamily{\scriptsize{Mobile games}}}     
 \rhead{\sffamily{\scriptsize{Siarhei Tsikhan}}}  
\rfoot{\centering\thepage}                                                       



\title{\vspace{-1em}\textbf{\sffamily{ Mobile games}}}   
\author{                                                 
    \large {Siarhei Tsikhan} \\                         
    \normalsize{Fakulta informatiky a informačných technológií}\\              
    \normalsize {Slovenská technická univerzita v Bratislave}\\         
}
\date{}                                                               
 
 

\begin{document}                   
\small          
			
\maketitle                         
				
\thispagestyle{fancy}     

\begin{multicols}{1}


\section{Introduction}
While the computer and console video games used to be the defining forms for digital gaming, mobile games have arguably established themselves as the contemporary, dominant sites for digital play. This development has been rather quick – only taking place in a timespan of couple of decades – and transformative for both the contents and practices of gaming, as well as to the associated gaming technologies and business models. Simultaneously, the detachment of digital play from the fixed location in homes and dedicated video gaming arcades has spread games and play into everyday lives in an unprecedented manner.

 This pervasive character and ease of access in mobile gaming is connected with several social and cultural changes: suddenly almost everyone seems to own a capable gaming device, and while there has been celebration of mobile gaming helping games to “go mainstream”, there has also been concerns (both by gamers and non-gamers alike) that all the associated changes have not been only for the good. \cite{Old}. 


\section{Mobile Game Evolution}
\subsection{Mobile games in the past}
The mobile gaming industry is booming. Smartphones have become more affordable and the industry has overtaken PC and console as the biggest mobile gaming category.
Games ventured from board games to PC games to now mobile games. One of the popular family game of snakes and ladder which kids enjoyed can now be played on the phone in a multiple mode.
 
It started in 1990’s with Nokia’s hugely successful game, Snake, which came as a feature on the phone itself. Users didn’t need to search or download a thing. Today, getting your hands on a mobile game requires searching, or browsing by categories, sometimes spending money, and then downloading it on your mobile device.\cite{Bad}. 

Japan was the first country to commercialize mobile games. This spread through to Asia, Europe, North America and eventually the rest of the world where smartphones were available. It was only when Apple launched its app store that the concept of gaming changed dramatically. Screens had color, consumer behavior changed, and almost everyone with a smartphone began downloading games. As phones got smarter, so did the games.

\subsection{Mobile games in the modern world}
The last few years have brought with them changes to the mobile gaming industry in that we can now play games that were once restricted to mobile, on desktop. Android emulators allow users to play almost any game on PC and Mac in exactly the same way as they would on mobile.

Having a larger screen to play on is transforming the mobile gaming world. We are no longer limited to mobile screens when playing mobile games.\cite{Now}. 

The significant rise in mobile gaming has drawn advertisers, as it presents an opportunity to easily reach millions of users. Because of the high demand for free mobile games to download today, in-app advertising platforms like Appnext have become a way for app developers to connect with each other in order to advertise their game within other games. The technique of not letting the player be able to move on in a game, for example, needing to wait for a video to complete or a task to finish provided advertisers with a recipe for success.

Android games are one of the largest markets with an audience of about 2 billion users. Among the most famous products are Angry Birds and Pokemon Go.

Android-based applications statistics and features:
\begin{enumerate}
\item Only 3\% of customers donate to mobile games;
\item 80\% of store sales are in game software;
\item In 2010, the volume of mobile game markets was estimated at about 33 billion U.S. dollars;
\item in 2017, the corresponding figure has passed 50 billion (about 43\% of the entire global gaming market);
\item the main consumers are women 35-45 years old.
\end{enumerate}

\section{Why are Mobile Games so Popular?}
Smartphones are a device almost everyone has, irrespective of age and station. Whether it’s your granddad or your five year old sibling, they can find and download games and immediately start playing, no matter where they are. According to AppAnnie, players downloaded 82.98 billion mobile games in 2021 and 64 percent of people who play mobile games do so on a daily basis. In 2020, gaming apps were also the app categories reporting the longest session lengths per user. 

Mobile games are easily accessible to all, extremely convenient to download and play as opposed to setting up a PC or console and their peripherals, and the best part is, you can enjoy them anywhere, anytime. You can download your favourite game from app stores and play it when you are waiting for a taxi or your restaurant order. When it comes to developers and publishers, it is cheaper to make a mobile game and since most mobile games released employ a F2P business model, the barriers to entry are negligible. These days, mobile games are also known for their creativity and console-quality titles. There are already joysticks and controllers specially tailored for mobile phones and in the future, we could very well see titles available for the pc or console playable on mobile.

Of course, once mobile games were proved to be something the public loved, companies had to figure out a way to make money from them without annoying their audience (too much.)

\section{The Future of Mobile Gaming}
In the past few years, mobile gaming has received a lot of attention from companies who’re looking to buff up their portfolios and profits. Since launch, Pokémon Go has generated 2.3 billion in revenue and Fortnite has amassed some 250 million players. Perhaps in an attempt to replicate such success, the first substantial investments in mobile gaming came from those who already had a stake in the industry. Tencent invested 90 million in Pocket Gems, gaming powerhouse Supercell invested 5 million in mobile game studio Redemption Games, Boom Fantasy raised 2 million from ESPN and the MLB, and Gamelynx raised 1.2 million from several investors. 

The first investment from an arguably old-school enterprise came from Goldman Sachs, who invested 200 million in hyper-casual mobile gaming studio Voodoo. In July 2018, private equity firm KKR bought a 400 million minority stake in AppLovin and a year later, Blackstone announced their plan to acquire mobile ad-network Vungle for a reported 750 million. Take Two and Microsoft are also foraying into the mobile platform with their acquisitions. The former’s acquisition of Zynga lets them get their hands on cash generators like Farmville while also letting them learn how to make successful F2P mobile titles from a proven expert. \cite{Good}. 

Cross-platform play is one of the hardest to achieve features a game can have. The differences in software architecture which different platforms such as Xbox, PlayStation, and PC have make creating ports for all of them simultaneously an onerous undertaking. It might also lead to one particular platform being favoured over others, such as in the case of Fortnite, where although there is a large player base for both console and mobile devices, the mechanics are easier on PC. For competitive play, Epic Games decided to create separate tournaments and lobbies to overcome these issues, but if you’re a console player looking to play with your PC friends, odds are your game will suddenly jump in difficulty. However, games like Genshin Impact have proven that cross-platform games when done right can increase your net revenue while also increasing your player base significantly. Perks like this can make cross-platform the new standard games aspire to, and increase mobile gamings’ already massive audience. 


\section{Conclusion}
Video games allow you to get rid of your thoughts and give your brain a much-needed refresh. Many games require abstract ideas and high-level thinking to break records. A good video game session on your cell phone can lift your spirits.  

                                   
\bibliography{literatura}
\bibliographystyle{plain}               

\end{multicols}				
\end{document}                                     